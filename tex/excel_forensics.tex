\documentclass[letterpaper, 12pt]{article}
\input{tex/_preamble}
\begin{document}
\flushleft\includegraphics[width=0.5\textwidth]{img/home/241017_final_logo_mockup.png}

\cosigsectionnew{Data forensics using Micosoft Excel}{15 September 2025}
%\textit{Last updated: 15 September 2025}

Scientific data is often stored in a tabular format. File formats for storing tabular data include:

\begin{itemize}
    \setlength\itemsep{-0.5em}
    \item \texttt{.csv}: comma-separated values, text-based
    \item \texttt{.tsv}: tab-separated values, text-based
    \item \texttt{.xlsx}: modern Microsoft Excel workbook format, text-based (specifically, \href{https://en.wikipedia.org/wiki/XML}{XML}-based)
    \item \texttt{.xlsm}: modern Excel workbook format with \href{https://support.microsoft.com/en-us/office/quick-start-create-a-macro-741130ca-080d-49f5-9471-1e5fb3d581a8}{macros} enabled, text-based
    \item \texttt{.xlsb}: modern Microsoft Excel workbook format, binary
    \item \texttt{.xls}: older (pre-2003) Excel workbook format, binary
\end{itemize}

All of the above file formats can be opened in \href{https://en.wikipedia.org/wiki/Microsoft_Excel}{Microsoft Excel}, which can be a useful tool for spotting potential data irregularities. This guide covers tips for using Excel to investigate the integrity of data stored in a tabular format.

\subsection*{Investigating potential data manipulation using Excel's calculation chain file}

Manual data collection using spreadsheets usually involves adding row after row to a sheet, often within Excel.
This can be automated, such as by using software that collects form responses and appends each response as a row to a file.

Once data collection has occured, there is usually no reason to change the order of row, duplicates rows or edit row values out of the order they appear in the spreadsheet. If this has occurred, it does not necessarily imply that the data in the sheet was falsified or fabricated. However, in combination with other observations, it can be a strong indication that data manipulation has taken place. 

\pagebreak

This is an example of an edit made ``out of order'':

\begin{figure}[h!tbp]
    \centering
    \includegraphics[width=0.5\textwidth]{img/excel/example.png}
    \captionsetup{justification=centering}
    \caption*{An edit to cell B2 when cell B3 has already been written.}
\end{figure}

By examining the an Excel workbook's ``calculation chain'' file, you can confirm that certain cell values were edited. Be aware that it is possible to make edits out of order and disguise this operation so it is not detectable in the calculation chain file.

\subsubsection*{Applicability}

Calculation chain files are recoverable in the ``Office Open XML'' file format supported by Microsoft Excel 2007 and later (i.e., files with an \texttt{.xlsx} or \texttt{.xlsm} extension). 

Calculation chain files are not recoverable for the \texttt{.xls} file format, regardless of which version of Excel generated them.
They are also not recoverable in the binary \texttt{.xlsb} file format.
Calculation chain files may or may not be recoverable for Excel-format files generated by other software (e.g., at the time of writing,
Google Sheets does not generate a ``calculation chain'' file even when exporting to \texttt{.xlsx}).

\subsubsection*{Accessing an Excel workbook's internal files}

Say that you have an Excel workbook file you want to inspect:

\begin{figure}[h!tbp]
    \centering
    \includegraphics[width=0.2\textwidth]{img/excel/file_noext.png}
\end{figure}

First, configure your file explorer to show \emph{file extensions} if it's not already doing so (on Windows 11, this option is under \emph{View > Show > File name extensions}):

\pagebreak

\begin{figure}[h!tbp]
    \centering
    \includegraphics[width=0.5\textwidth]{img/excel/menu_show_exts.png}
\end{figure}

After this, you should be able to see \texttt{.xlsx} or \texttt{.xlsm} filename extension.
Select the file and choose to rename it (on Windows, press F2).
Instead of changing the display name of the file, change the filename extension from \texttt{.xlsx} or \texttt{.xlsm} to \texttt{.zip} (i.e., the file extension for a \href{https://en.wikipedia.org/wiki/ZIP_(file_format)}{ZIP} archive). You may need to accept a prompt confirming you really want to change the filename extension.


\hspace{0.1\linewidth}
\begin{minipage}{0.2\linewidth}
    \includegraphics[width=1\textwidth]{img/excel/file_withext.png}
\end{minipage}
\hspace{0.1\linewidth}
\begin{minipage}{0.2\linewidth}
    \includegraphics[width=1\textwidth]{img/excel/file_rename.png}
\end{minipage}
\hspace{0.1\linewidth}
\begin{minipage}{0.2\linewidth}
    \includegraphics[width=1\textwidth]{img/excel/file_zip.png}
\end{minipage}

You can now inspect the contents of the Excel file by opening this ZIP archive!
Indeed, the modern Microsoft Office file formats (like \texttt{.xlsx}, \texttt{.pptx} and \texttt{.docx}) are all ZIP archives in disguise.

For more details, consult the \href{https://ecma-international.org/publications-and-standards/standards/ecma-376/}{ECMA-376 standard} that defines Office Open XML formats.

\subsubsection*{Locating the calculation chain file}

To detect out-of-order edits, we will use the ``calculation chain'' file, which records the order of all formulas.
It instruct Excel on which order to run the formulas in the spreadsheet so that Excel does not have to recompute the order every time the file is opened.

\pagebreak

Inside your ZIP archive, locate the \texttt{xl} folder, and within it, the \texttt{calcChain.xml} file. Open it in any text viewer you'd like (e.g., Notepad), and you should see contents like the following:

\texttt{<calcChain xmlns="...">\\
~~<c r="B2" i="1" l="1"/>\\
~~<c r="B3" i="1"/>\\
~~<c r="B1" i="1"/>\\
</calcChain>}

This file contains one entry for each cell with a formula in reverse execution order (i.e., the last-executed formula appears first). The important part is the \texttt{r} attribute of each \texttt{c} element, which indicates the cell or range over which the formula is applied.

In this example, the first formula entered by the user was in cell \texttt{B1}, then \texttt{B3}, then \texttt{B2},
which clearly shows an ``out of order'' edit. In this case, it is likely that the user initially input the formulas in order then swapped the order of the 2nd and 3rd rows.

The \texttt{i} attribute of each \texttt{c} element indicates the ID of the sheet the cell is in.
In the example above, all cells are in the same sheet, but it can differ for multi-sheet spreadsheets, for instance:

\texttt{<calcChain xmlns="...">\\
~~<c r="A1" i="2" l="1"/>\\
~~<c r="B1" i="1"/>\\
</calcChain>}

To map each sheet ID to the corresponding sheet, open the \texttt{workbook.xml} file in the \texttt{xl} folder of the archive,
which, among other things, contains a \texttt{sheets} element whose children define which ID corresponds to which sheet name:

\texttt{<sheets>\\
~~<sheet name="First sheet" sheetId="1" r:id="rId1"/>\\
~~<sheet name="Other sheet" sheetId="2" r:id="rId2"/>\\
</sheets>}

Other attributes, like those related to sub-chains and parallel computations, are described in detail \href{https://learn.microsoft.com/en-us/dotnet/api/documentformat.openxml.spreadsheet.calculationcell}{here}.

\subsection*{Spotting duplicate values}

You can visually distinguish duplicate values in a range of cells by using Excel's conditional formatting options.
Specifically, under \emph{Home > Conditional Formatting > Highlight Cells Rules > Duplicate Values},
you can pick the format of your choice for duplicated values.
More details are available \href{https://support.microsoft.com/en-us/office/find-and-remove-duplicates-00e35bea-b46a-4d5d-b28e-66a552dc138d}{in the official Microsoft documentation}.

You can also use Excel's \emph{Home > Conditional Formatting > Color Scales} feature to highlight
duplicates of values in a specific columns (see \href{https://stackoverflow.com/a/53294784/3311770}{an example here}).

\subsection*{Example: the ``Data Falsificada'' case}

On their blog \href{https://datacolada.org/}{``Data Colada''}, scientists Uri Simonsohn, Leif Nelson and Joe Simmons investigated a case of potential data manipulation in \href{https://doi.org/10.1073/pnas.1209746109}{a 2012 article} co-authored by prominent behavioral scientists.

In their \href{https://datacolada.org/109}{June 2023 blog post}, they show how inspecting the calculation chain of the Excel file containing the raw data for one of the experiments in the article reveals signs of manipulation.

Specifically, in this experiment's recorded data, rows that should have been in one treatment condition according to their position in the spreadsheet
were originally in a different treatment condition according to the calculation chain. The moved rows contain numerical values for the dependent variable that were among the highest and lowest values for their condition, suggesting that the rows were moved to achieve a desired effect size when the treatment group and control groups were compared.

The article in question had previously \href{https://doi.org/10.1073/pnas.2115397118}{been retracted in 2021} for evidence of data fabrication in another experiment it reported, also discovered by Data Colada.

\subsection*{Additional resources}

\begin{itemize}
    \setlength\itemsep{-0.5em}
    \item \href{https://github.com/markusenglund/copy-paste-detective}{Markus Englund's ``copy paste detective'' tool}. This tool finds similar blocks of data within Excel files. It requires \href{https://www.npmjs.com/}{npm} to run.
    \item \href{https://ntnuopen.ntnu.no/ntnu-xmlui/bitstream/handle/11250/198656/EDidriksen.pdf}{``Forensic Analysis of OOXML Documents'' (2014)}
\end{itemize}

\begin{comment}
    For an example of similar blocks, see \href{https://pubpeer.com/publications/84805DBB9C70E24E7E6C4A80F60A43}{this PubPeer thread}.
\end{comment}

\end{document}